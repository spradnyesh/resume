%%%%%%%%%%%%%%%%%%%%%%%%%%%%%%%%%%%%%%%%%
% "ModernCV" CV and Cover Letter
% LaTeX Template
% Version 1.1 (9/12/12)
%
% This template has been downloaded from:
% http://www.LaTeXTemplates.com
%
% Original author:
% Xavier Danaux (xdanaux@gmail.com)
%
% License:
% CC BY-NC-SA 3.0 (http://creativecommons.org/licenses/by-nc-sa/3.0/)
%
% Important note:
% This template requires the moderncv.cls and .sty files to be in the same
% directory as this .tex file. These files provide the resume style and themes
% used for structuring the document.
%
%%%%%%%%%%%%%%%%%%%%%%%%%%%%%%%%%%%%%%%%%

%----------------------------------------------------------------------------------------
%   PACKAGES AND OTHER DOCUMENT CONFIGURATIONS
%----------------------------------------------------------------------------------------

\documentclass[11pt,a4paper,sans]{moderncv} % Font sizes: 10, 11, or 12; paper sizes: a4paper, letterpaper, a5paper, legalpaper, executivepaper or landscape; font families: sans or roman

\moderncvstyle{casual} % CV theme - options include: 'casual' (default), 'classic', 'oldstyle' and 'banking'
\moderncvcolor{blue} % CV color - options include: 'blue' (default), 'orange', 'green', 'red', 'purple', 'grey' and 'black'
\usepackage[scale=0.75]{geometry} % Reduce document margins

%----------------------------------------------------------------------------------------
%   NAME AND CONTACT INFORMATION SECTION
%----------------------------------------------------------------------------------------

\firstname{Pradnyesh} % Your first name
\familyname{Sawant} % Your last name

\title{Curriculum Vitae}
\mobile{(+91) 7208666924}
\email{spradnyesh@gmail.com}
% \homepage{http://spradnyesh.github.io}
\extrainfo{@spradnyesh (github, google, yahoo, twitter, facebook, skype)}

%----------------------------------------------------------------------------------------

\begin{document}

\makecvtitle % Print the CV title

%----------------------------------------------------------------------------------------
%   INTRODUCTION SECTION
%----------------------------------------------------------------------------------------

\section{Objective}
% To guide and mentor teams that build scalable and secure systems that are useful and/or entertaining to customers and also generate revenue for the company
% To work in the field of financial data analysis and prediction using data-mining and machine-learning technologies
To build scalable and secure systems that are useful and/or entertaining to customers and also generate revenue for the company
% To work on machine-learning and data-mining problems that provide interesting and useful insights which are useful for customers and also generate revenue for the company


%----------------------------------------------------------------------------------------
%   Key Skills and Achievements
%----------------------------------------------------------------------------------------
\section{Skillset}
\cvitem {Languages}{Clojure, Clojurescript, Common Lisp, Javascript, Java, PHP, Python, Bash, Latex}
\cvitem {Machine Learning}{Weka, Encog (Clojure with Java interop)}
\cvitem {Cloud/VPS}{Redhat OpenShift, Linode}
\cvitem {Frameworks}{Luminus (Clojure), Restas (Common Lisp), Spring (J2EE), Symfony (PHP)}
\cvitem {Libraries}{Om, Reagent, Bootstrap}
\cvitem {Web-Servers}{Nginx, Hunchentoot (Common Lisp), Apache, JBoss}
\cvitem {Databases}{MongoDB, PostgreSQL, MySQL, SQLite3}
\cvitem {Agile}{Scrum, TDD}
% \cvitem {Tools}{spacemacs, org-mode, CIDER for clojure}

\section{Achievements}
\cvitem {}{
\begin{itemize}
  \item Able to register and setup a software development company and build product using cutting edge machine learning technologies
  \item Won the best employee award in the ``Store Development Group'' in 2nd half of 2013 at Samsung Electronics, South Korea
  \item Team won the Superstar award, which is the most coveted award at Yahoo! worldwide, for the EntertainmentLite project
  \item Built http://maktoob.omg.yahoo.com which was the 2nd completely homegrown Arabic (RTL) website in Yahoo!
  \item Team awarded the Odyssey award for excellence in engineering efforts at Yahoo! for the http://360plus.yahoo.com project
  \item Topped in 3 of 4 semesters and ranked 2nd in MTech at IIT Guwahati
  \item Awarded scholarship by Philips Research Labs, Bangalore for research in IIT Guwahati
  \item Selected for sponsored project for Network Simulator in Java, sponsored by NICT, Japan, while studying at IIT Guwahati
\end{itemize}}

%----------------------------------------------------------------------------------------
%   WORK EXPERIENCE SECTION
%----------------------------------------------------------------------------------------
\newpage{}
% \section{Work Experience}
\section{Gryffin Software Development LLP (Mumbai, India)}
\cventry {Apr '2012 -- Present}{Solo-preneur (http://stox.gryff.in)}{}
         {}{Technology: Clojure + MongoDB + OpenShift + Bootstrap, Weka/Encog}
         {The project involves stock price predction and portfolio creation using latest cutting edge machine learning technologies and functional programming techniques. I am successfully able to achieve about 95\% prediction accuracy for next day prices using the Weka toolkit. As a solo-preneur, I have to look at all aspects of the project, both technical like ideation, implementation and testing, deployment, and others like registering the company, finance, advertising, etc.}

\section{Open Source Contributions}
\cventry {Apr '2012 -- Present}{https://github.com/spradnyesh}{}{}{}
         {\begin{itemize}
           \item golbin: a blogging system (including both editorial and frontend websites) written in common-lisp and having google-ads integration (view archives at https://web.archive.org/web/20141217075454/http://golb.in/)
           \item cl-web-utils: a library of utilities that emerged out of golbin development. It has lots of utility functions for encryption, cron, datetime, database, email, google-ads integration, html, http, js, l10n (localization), memcache, pagination, images, and more
           \item Contribution to existing projects
             \begin{itemize}
             \item encog-java-core
             \item encog-java-workbench
             \item clojure-cartridge
             \item luminus-template
             \item stumpwm-contrib
             \end{itemize}
           \item hobby projects: clj-encog-examples, mc-sentimeter, cljs-games, bulk-trader, cl(j)-project-euler
         \end{itemize}}

\section{Senior Engineer (S5 -- level 5), Samsung Electronics (Suwon, S. Korea)}
\cventry {Apr '2012 -- Mar '2014}{In-App-Purchase Server for Samsung Apps}{}
         {Team size: 20}{Technology: J2EE}
         {The project involved from scratch re-write of the ``In App Purchase'' part of Samsung Apps
server using J2EE technologies (Spring framework, JBoss, Informatica Power Center for
ETL, etc). I functioned as the Engineering Manager for this project and the sole
representative from HQ responsible for managing the outsourced team (Samsung Research,
Poland (SRPOL)) and releasing the server with good quality (around 900+ CI powered unit tests) on schedule. Some of the biggest challenges for me working in this project were finding a
common ground between 2 radically different and opposing cultures (Korean and Polish),
and also drawing up the business and logic requirements from an old and undocumented, but
in service, codebase. I was also responsible for the smooth migration of codebase and
knowledge from SRPOL to SDC, Korea towards the end of SRPOL contract; and later for
guiding and managing SDC team members.}

\section{Principal Engineer, Yahoo! SDC (Bangalore, India)}
%% \cventry {Jul '07 -- Mar '12}{Yahoo! Frontpage}{http://yahoo.com}
%%          {Team size: 10}{Technology: LAMP}
%%          {}
\cventry {Dec '2011 -- Mar '2012}{Cars}{http://uk.cars.yahoo.com/motorshow/frankfurt/}
         {Team size: 4+3}{Technology: LAMP}
         {I worked as the Engineering Manager in this project and had 10 people working as my team.
We were able to release 5 Cars websites (for DE, ES, FR, GB and IT) end-to-end within about 7
weeks including feeds processing. The websites were a subset of the Autos US version (below).}
\cventry {Mar '2011 -- Dec '2011}{Autos}{http://autos.yahoo.com/news/}
         {Team size: 4+2}{Technology: LAMP}
         {This project was an attempt to add a magazine like look, using articles, photos, videos, etc., to
the otherwise technical content of autos.yahoo.com in order to increase user stickiness. For
this (and other properties currently being upgraded similarly), a central team was setup, led
and managed by me, that interacts with the core property team for API level support and with
the platforms teams to build pages and custom modules to develop and deploy the magazine
section of the website.}
\cventry {Jan '2010 -- Mar '2011}{EntertainmentLite}{http://id.omg.yahoo.com}
         {Team size: 7+3}{Technology: LAMP}
         {This project is a collection of 6 entertainment related websites (*.omg.yahoo.com built for
Indonesia, Brazil, India, Middle-East, Mexico and Espanol). The admirable thing about this
project is that it was built in parallel with the platform, which supported us and also another
property (LifestyleLite), in just 5 months. In this project I worked on the performance
analysis and continuous integration (CI) of both EntertainmentLite and LifestyleLite and
made architectural decisions for the common modules and website specific components. I was
responsible for managing the sprints and ensuring that the team delivered quality code within
the deadlines. In addition, I managed all inter-team collaboration, feeds related issues and
editorial training. I was also largely responsible for the RTL-ization of the middle-east website.}
\cventry {Mar '2009 -- Dec '2009}{Digu}{http://digu.yahoo.com}
         {Team size: 10+6}{Technology: LAMP}
         {One of the key features about this project was that it was collaboration between the
Bangalore and Taiwan teams. One of the most important lessons I learned in this
project was about making things work across teams differing in not only cultures but also
work practices, about finding the good things in a team and encouraging the others to follow
them.}
\cventry {Jul '2008 -- Mar '2009}{Spotm}{http://in.spotm.yahoo.com}
         {Team size: 4+4}{Technology: LAMP}{}
\cventry {Nov '2007 -- Jul '2008}{Vn Blogs}{http://360plus.yahoo.com}
         {Team size: 2}{Technology: LAMP}{}
\cventry {Jul '2007 -- Oct '2007}{Listit}{http://in.listit.yahoo.com}
         {Team size: 4+2}{Technology: LAMP}{}
\cventry {Jul '2007 -- Mar '2012}{Miscellaneous}{}{}{}
         {I worked as part of the Globalization and FE Performance expert teams, and was a member
of the technical trainers team (python, globalization).}

\section{Associate Systems Analyst, Nse.iT, (Mumbai, India)}
\cventry {2003 -- 2005}{Parallel RISk Management (PRISM)}{}{}{Technology: C, Java}{}

%----------------------------------------------------------------------------------------
%   EDUCATION SECTION
%----------------------------------------------------------------------------------------

\section{Academics}
\subsection{Education}
\cventry{2005 -- 2007}{Masters of Technology (M. Tech.)}{IIT Guwahati}{Guwahati}{\textit{9.34 CPI}}{}
\cventry{1999 -- 2003}{Bachelors of Engineering}{Mumbai University}{Mumbai}{\textit{64.36\%}}{Information Technology}
\subsection{M. Tech. Project}
\cventry{2006 -- 2007}{Fault Tolerance in a Server Cluster Backend for Thin Client Environment}{}{}{}{}

\end{document}
